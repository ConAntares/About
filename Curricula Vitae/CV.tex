% Academic Curriculum Vitae
% Copyright 2018 Lu Niu
% Email: LukeNiu@outlook.com
% GitHub: https://github.com/ConAntares

\documentclass[12pt,a4paper,utf8]{report}
\usepackage[dvipsnames]{xcolor}
\usepackage{datetime}
\usepackage{fancyhdr}
\usepackage{geometry}
\usepackage{titlesec}
\usepackage{framed}
\usepackage{color}
\usepackage{tabto}
% \usepackage{droid}    % serif font
\usepackage{crimson}    % serif font
\usepackage{helvet}     % sans serif font

\geometry{left=2.0cm,right=2.0cm,top=2.0cm,bottom=2.0cm}
\pagestyle{fancy}
\definecolor{shadecolor}{rgb}{0.0,0.5,0.8}
\definecolor{subtitlecolor}{rgb}{0.5,0.8,0.9}
\linespread{0.9}

\newcommand{\namefont}[1]{{\normalfont\bfseries\Huge{#1}}}
\newcommand{\myname}{Lu Niu}
\newcommand{\mydegree}{PhD Candidate in Physics}
\newcommand{\mywork}{\quad}%{Teaching Assistant on Analytical Mechanics}
\newdateformat{monthyeardate}{\monthname[\THEMONTH] \THEYEAR}


\fancyhead[C]{
    \colorbox{shadecolor}{\raisebox{0pt}[\height][0pt]{
        \textcolor{white}{\textsf{\leftline{
            \quad CURRICULUM VITAE
}}}}}}

% \fancyfoot[C]{\thepage}

\renewcommand\headrulewidth{0.0pt}

\begin{document}

\begin{minipage}[t]{8cm}
    \begin{center}
        \vspace{0.0cm}\namefont{\myname}\\[0.1cm]
    \end{center}
\end{minipage}
\begin{minipage}[t]{8cm}
    \vspace{0.0cm}
    {\em{\mydegree}}\par
    {\em{\mywork}}\par
\end{minipage}

\begin{minipage}[t]{8cm}
    \vspace{0.4cm}
    Condensed Matter Theory Group \\
    School of Physics \\
    The University of Sydney \\
    Sydney, NSW 2006, Australia\\
\end{minipage}
\begin{minipage}[t]{1.5cm}
    \vspace{0.4cm}
    Email: \\
    Web: \\
    GitHub: \\
    Phone: \\
\end{minipage}
\begin{minipage}[t]{8cm}
    \vspace{0.4cm}
    Luke.Niu@sydney.edu.au\\
    https://github.com/ConAntares \\
    https://github.com/ConAntares \\
    +86 13811152301, +61 0451116402\\
\end{minipage}
\vspace{-0.2cm}

\begin{minipage}[t]{16cm}
    \colorbox{subtitlecolor}{\raisebox{0pt}[10.0pt][1.0pt]{
        \textcolor{white}{\textsf{\leftline{
            \quad Personal Information
    }}}}}
\end{minipage}\par
\vspace{0.2cm}
    \begin{minipage}[t]{4cm}
        \qquad \textbf{Gender:}\par
        \qquad \textbf{Day of Birth:}\par
        \qquad \textbf{Place of Birth:}\par
        \qquad \textbf{Nationality:}\par
        \qquad \textbf{Office:}\par
    \end{minipage}
    \begin{minipage}[t]{12cm}
        Male \par
        May 12, 1993 \par
        Huairou District, Beijing, P.R.China \par
        The People's Republic of China \par
        Room 443, School of Physics A28 \par
    \end{minipage}\par
\vspace{0.4cm}

\begin{minipage}[t]{16cm}
    \colorbox{subtitlecolor}{\raisebox{0pt}[10.0pt][1.0pt]{
        \textcolor{white}{\textsf{\leftline{
            \quad Research Interests
    }}}}}
\end{minipage}\par
\vspace{0.2cm}\hspace{0.5cm}
    \begin{minipage}[t]{15.0cm}
        {\textbf{Quantum Computation:}}\par
        \quad Quantum circuit. \par
    \vspace{0.2cm}
        {\textbf{Ab Initio:}}\par
        \quad The surface effect of metal nano-particles.
    \end{minipage}\par
\vspace{0.4cm}

\begin{minipage}[t]{16cm}
    \colorbox{subtitlecolor}{\raisebox{0pt}[10.0pt][1.0pt]{
        \textcolor{white}{\textsf{\leftline{
            \quad Education
    }}}}}
\end{minipage}\par
\vspace{0.2cm}\hspace{0.5cm}
\begin{minipage}[t]{15.0cm}
    {{\textbf{The University of Sydney}}}\hfill
    {\em{Jul. 2019 -- Present}}\par\vspace{0.1cm}
    {\qquad\em{PhD of Science in Physics}}\par
\end{minipage}\par
\vspace{0.1cm}
    \begin{minipage}[t]{4cm}
        \qquad \textbf{Subject:}\par
        \qquad \textbf{Thesis:}\par
        \qquad \par
        \qquad \textbf{GPA:}\par
        \qquad \textbf{Supervisor:}\par
    \end{minipage}
    \begin{minipage}[t]{11cm}
        Plasmonics\par
        Quantum Computation Logic Circuits Realization Based on Plasmon Effects.\par
        Expected\par
        Catherine Stampfl\par
    \end{minipage}\par
\vspace{0.2cm}\hspace{0.5cm}
\begin{minipage}[t]{15.0cm}
    {{\textbf{University of Science and Technology Beijing}}}\hfill
    {\em{Sep. 2015 -- Jun. 2017}}\par\vspace{0.1cm}
    {\qquad\em{MPhil of Science in Physics}}\par
\end{minipage}\par
\vspace{0.1cm}
    \begin{minipage}[t]{4cm}
        \qquad \textbf{Subject:}\par
        \qquad \textbf{Thesis:}\par
        \qquad \par
        \qquad \textbf{GPA:}\par
        \qquad \textbf{Supervisor:}\par
    \end{minipage}
    \begin{minipage}[t]{11cm}
        Atom and Molecular Physics\par
        Effect of External Field on the IV Characteristics through the Molecular Nano-junction.\par
        3.6\par
        Luxia Wang\par
    \end{minipage}\par
\vspace{0.4cm}

\begin{minipage}[t]{16cm}
    \colorbox{subtitlecolor}{\raisebox{0pt}[10.0pt][1.0pt]{
        \textcolor{white}{\textsf{\leftline{
            \quad Expertise and Technical Strengths
    }}}}}
\end{minipage}\par
\vspace{0.2cm}
    \begin{minipage}[t]{4cm}
        \qquad \textbf{Programming:}\par
        \qquad \textbf{Software:}\par
        \qquad \textbf{Expertise:}\par
        \qquad \par
        \qquad \textbf{Language:}\par
    \end{minipage}
    \begin{minipage}[t]{14cm}
        C/C++, Fortran, Python, Julia (Main), Gnuplot, \LaTeX \par
        Linux, Git, TensorFlow, VASP, Octopus \par
        Quantum Optics, Quantum Plasmonics, Computational Physics, \\Mathematical Analysis, Algorithms \par
        Mandarin Chinese (Native), English \par
    \end{minipage}\par
\vspace{0.4cm}

\begin{minipage}[t]{16cm}
    \colorbox{subtitlecolor}{\raisebox{0pt}[10.0pt][1.0pt]{
        \textcolor{white}{\textsf{\leftline{
            \quad Work Experience
    }}}}}
\end{minipage}\par
\vspace{0.2cm}\hspace{0.5cm}
\begin{minipage}[t]{15cm}
    {{\textbf{University of Science and Technology Beijing}}}\par
    \quad Teaching Assistant on Analytical Mechanics \hfill
    {\em{Spring, 2017 and Spring, 2016}}\par
    \quad Teaching Assistant on College Physics\hfill
    {\em{Autumn, 2016 and Autumn, 2015}}\par
\end{minipage}\par
\vspace{0.4cm}

\newpage
\begin{minipage}[t]{16cm}
    \colorbox{subtitlecolor}{\raisebox{0pt}[10.0pt][1.0pt]{
        \textcolor{white}{\textsf{\leftline{
            \quad Research Experience
    }}}}}
\end{minipage}\par
\vspace{0.2cm}\hspace{0.5cm}
\begin{minipage}[t]{15cm}
    {\textbf{Effect of External Field on the IV Characteristics through the Molecular Nano-junction}}\par
    \qquad\qquad\qquad\qquad\qquad\qquad{\em{Sep. 2016 -- Jun. 2017 @ USTB, Beijing, P.R.China}}\par
    \quad This research involves molecule physics. we analyzed the steady current between two electrodes under distinct bias voltages, and studied transient current under Gaussian pulse with different widths; we established the physical model of Molecular junction with external fields which could produce coupling with the molecule.\par
    \vspace{0.2cm}
    {\textbf{Plasmon-Enhanced Heterogeneous Electron Transfer with Continuous Band Energy Model}}\par
    \qquad\qquad\qquad\qquad\qquad\qquad{\em{Apr. 2016 -- Mar. 2017 @ USTB, Beijing, P.R.China}}\par
    \quad We calculated the Plasmon-Enhanced heterogeneous electron transfer in semiconductor continuous model with the master equation. And simulated the physical model and conducted the scientific calculation.\par
    \vspace{0.2cm}
    {\textbf{Molecular Emission Spectrum of Combined System and its Fourier Analysis}}\par
    \qquad\qquad\qquad\qquad\qquad\qquad{\em{Dec. 2015 -- Apr. 2016 @ USTB, Beijing, P.R.China}}\par
    \quad We probed into the emission spectrum of molecular with Fourier analysis. And built the equations set which describes the physical process of the molecule system excitation in the quantization radiation field.\par
\end{minipage}\par
\vspace{0.4cm}

\begin{minipage}[t]{16cm}
    \colorbox{subtitlecolor}{\raisebox{0pt}[10.0pt][1.0pt]{
        \textcolor{white}{\textsf{\leftline{
            \quad Publications
    }}}}}
\end{minipage}\par
\vspace{0.2cm}\hspace{0.5cm}
\begin{minipage}[t]{15cm}
    {\textbf{\em{2018}}}\par
    \quad Lu Niu, Luxia Wang*; {\em{Effect of External Field on the I-V Characteristics through the Molecular Nano-junction}} (in Chinese); Acta Physica Sinica, 67, 027304 (2018).\par
    \vspace{0.2cm}
    {\textbf{\em{2017}}}\par
    \quad Dandan Zhao, Lu Niu, Luxia Wang*;{\em{Plasmon Enhanced Heterogeneous Electron Transfer with Continuous Band Energy Model}}; Chemical Physics, 493 (2017) 194-199.\par
\end{minipage}\par
\vspace{0.4cm}

\begin{minipage}[t]{16cm}
    \colorbox{subtitlecolor}{\raisebox{0pt}[10.0pt][1.0pt]{
        \textcolor{white}{\textsf{\leftline{
            \quad References and Activites
    }}}}}
\end{minipage}\par
\vspace{0.2cm}\hspace{0.5cm}
\begin{minipage}[t]{15cm}
    {\textbf{\em{2016}}}\par
    \quad Nov. 14 -- Nov.18, Beijing, The 2nd Joint Workshop on Condensed Matter Science, Peking University \& IMPRS. @ PKU, Beijing, P.R.China.\par
    \vspace{0.2cm}
\end{minipage}\par
\vspace{0.4cm}

\begin{minipage}[t]{16cm}
    \colorbox{subtitlecolor}{\raisebox{0pt}[10.0pt][1.0pt]{
        \textcolor{white}{\textsf{\leftline{
            \quad Awards and Honors
    }}}}}
\end{minipage}\par
\vspace{0.2cm}\hspace{0.5cm}
\begin{minipage}[t]{15cm}
    % {\textbf{\em{2016}}}\par
    % \quad First-class Scholarship, University of Science and Technology Beijing.\par
    \vspace{0.2cm}
\end{minipage}\par
\vspace{0.4cm}

\begin{center}\vspace{1.0cm}
    Updated \monthyeardate\today
\end{center}

\end{document}